\documentclass[11pt]{article}
\usepackage[margin=1in]{geometry}
\usepackage{amsmath, amssymb}
\usepackage{graphicx}
\usepackage{hyperref}

\title{\textbf{Physics-Informed Neural Networks Assignment}}
\author{CE397 and CSE393: Scientific Machine Learning}
\date{}
\begin{document}
	\maketitle
	\hrulefill
	
	\section{1D Steady-State Heat Equation}
	Consider the one-dimensional steady-state heat conduction problem with a heat source on a rod of unit length. The governing equation is:
	
	\begin{equation*}
		\frac{d^2T}{dx^2} + \frac{q(x)}{\kappa} = 0, \quad x \in [0,1]\,,
	\end{equation*}
	with homogeneous Dirichlet boundary conditions:
	\begin{equation*}
		T(0) = T(1) = 0\,.
	\end{equation*}
	Given a thermal diffusivity constant: $\kappa = 0.5$
	and Heat source term: $q(x) = 15x - 2$.
	
	\begin{enumerate}
		\item[\textbf{a)}] Design a physics-informed neural network with:
		\begin{itemize}
			\item Input dimension: 1 (spatial coordinate $x$)
			\item Three hidden layers with 32 neurons each
			\item Hyperbolic tangent activation functions
			\item Output dimension: 1 (temperature $T$)
		\end{itemize}
		
		\item[\textbf{b)}] Construct the physics-informed loss function $\mathcal{L}_{Physics}$ that enforces the PDE.
		
		\item[\textbf{c)}] Write the boundary loss function $\mathcal{L}_{BCs}$.
		
		\item[\textbf{d)}] Train the network using:
		\begin{itemize}
			\item $N_f = 100$ equidistant collocation points
			\item Total loss $\mathcal{L} = \mathcal{L}_{BCs} + \mathcal{L}_{Physics}$
			\item Adam optimizer with learning rate $10^{-3}$
			\item 10,000 epochs
		\end{itemize}
		
		\item[\textbf{e)}] Derive the analytical solution and compare it with the numerical solution. Plot both solutions and compute the relative $L^2$ error.
	\end{enumerate}
	
	\section{Loss Function Design}
	Investigate different formulations of the loss function:
	
	\begin{enumerate}
		\item[\textbf{a)}] Implement adaptive loss weighting:
		\begin{equation*}
			\mathcal{L} = \alpha(t)\mathcal{L}_{BCs} + \beta(t)\mathcal{L}_{Physics}
		\end{equation*}
		where $\alpha(t)$ and $\beta(t)$ are dynamic weights.  Compare with the original fixed-weight formulation. 
	\end{enumerate}
	
	\section{Variational Formulation [Optional for CE397]}
	Implement and analyze a variational (weak) form of the problem:
	
	\begin{enumerate}
		\item[\textbf{a)}] Derive the weak form by multiplying the PDE by test functions $\phi(x)$ and integrating:
		\begin{equation*}
			\int_0^1 \frac{dT}{dx}\frac{d\phi}{dx}dx = \int_0^1 \frac{q(x)}{\kappa}\phi dx
		\end{equation*}
		
		\item[\textbf{b)}] Implement this formulation using neural networks
		and compare with the strong form implementation 
	\end{enumerate}
	

	\section{Data-Driven Cross-Section Identification}
	Consider the static bar equation where the displacement field $u(x)$ is known but the cross-sectional properties $EA(x)$ need to be identified:
	
	\begin{equation*}
		\frac{d}{dx}(EA(x)\frac{du}{dx}) + p(x) = 0, \quad x \in [0,1]
	\end{equation*}
	
	Given:
	\begin{itemize}
		\item Displacement field: $u(x) = \sin(2\pi x)$
		\item Distributed load: $p(x) = -2(3x^2 - 2x)\pi \cos(2\pi x) + 4(x^3 - x^2 + 1)\pi^2 \sin(2\pi x)$
		\item Domain: $x \in [0,1]$
		\item Boundary conditions: $u(0) = u(1) = 0$
	\end{itemize}
	
	\begin{enumerate}
		\item[\textbf{a)}] Design a physics-informed neural network to identify $EA(x)$ with:
		\begin{itemize}
			\item Input dimensions: 2 (spatial coordinate $x$ and displacement $u$)
			\item Three hidden layers with 20 neurons each
			\item Hyperbolic tangent activation functions
			\item Output dimension: 1 (stiffness $EA$)
		\end{itemize}
		
		\item[\textbf{b)}] Formulate the physics-informed loss function that enforces the differential equation.
		
		\item[\textbf{c)}] Train the network using:
		\begin{itemize}
			\item $N = 100$ uniformly distributed training points
			\item Adam optimizer with learning rate $10^{-3}$
			\item 5,000 epochs
		\end{itemize}
		
		\item[\textbf{d)}] Compare the identified $EA(x)$ with the analytical solution:
		\begin{equation*}
			EA(x) = x^3 - x^2 + 1
		\end{equation*}
		
		\item[\textbf{e)}] Study the influence of noise in the displacement measurements by adding Gaussian noise with a standard deviation of 0.01.
	\end{enumerate}
	
\end{document}