\documentclass[11pt]{article}
\usepackage[margin=1in]{geometry}
\usepackage{amsmath, amssymb}
\usepackage{graphicx}
\usepackage{hyperref}

\title{\textbf{Graph Neural Network Simulator Assignment}}
\author{CE397 and CSE393: Scientific Machine Learning}
\date{}
\begin{document}
	\maketitle
	\hrulefill
	\section{Implementing Loss Functions for Particle Simulation}
	The notebook \texttt{gnn.ipynb} implements a Graph Neural Network-based Simulator (GNS) to predict the motion of particles in a "WaterDrop" simulation. The model is trained to predict the normalized acceleration of each particle. The standard training loop uses a simple Mean Squared Error loss on this predicted acceleration.
	
	In this assignment, you will implement and compare two different loss functions: one based on the predicted acceleration (the default) and a new one based on the predicted position after one step of simulation.
	
	\begin{enumerate}
		\item[\textbf{a)}] \textbf{Acceleration-based Loss (Default)}
		
		The default training procedure in the \texttt{train} function (cell 63) uses \texttt{torch.nn.MSELoss()}. This computes the mean squared error between the model's direct output (predicted normalized acceleration, $\mathbf{y}_{\text{pred}}$) and the ground truth normalized acceleration ($\mathbf{y}_{\text{true}}$, available as \texttt{data.y}).
		
		Write the mathematical formula for this loss, $\mathcal{L}_{accel}$, using $N$ for the number of particles in the batch.
		
		\item[\textbf{b)}] \textbf{Position-based Loss (New Implementation)}
		
		A loss function based on the predicted particle \textit{position} at the next timestep, $\mathbf{p}^{t+1}$, may provide a more physically relevant training signal. To implement this, you must modify data processing and implement a position loss.

		\item[\textbf{c)}] \textbf{Train and Compare}
		
		Modify the \texttt{train} function to use your new position-based loss function, $\mathcal{L}_{pos}$. Run the training for at least one epoch (\texttt{params["epoch"] = 1}) and report the final "One Step MSE" from the evaluation. How does this value compare to the "One Step MSE" obtained when training with the original $\mathcal{L}_{accel}$?
		
		\textit{(Note: The \texttt{oneStepMSE} function itself calculates MSE on acceleration, so you may need to either adapt it or compare the \texttt{eval\_loss} values directly, ensuring your new loss function is also used during evaluation.)}

	\end{enumerate}
\end{document}
