\documentclass[11pt,fourier]{article}
\usepackage{amsmath, amssymb}
\usepackage{geometry}
\usepackage{fourier}
\usepackage[colorlinks=true,linkcolor=black,urlcolor=black]{hyperref}
\usepackage{array, booktabs, tabularx}
\usepackage{fancyhdr}
\usepackage{titlesec}
\usepackage{enumitem}

% Set page margins
\geometry{a4paper, margin=1in}

% Define header and footer
\pagestyle{fancy}
\fancyhf{}
\renewcommand{\headrulewidth}{0pt}
\fancyhead[L]{The University of Texas at Austin}
\fancyhead[R]{CE 397 Scientific Machine Learning}
\cfoot{\thepage}

% Title formatting
\titleformat{\section}
{\normalfont\Large\bfseries}{\thesection}{1em}{}
\titlespacing*{\section}{0pt}{12pt}{6pt}

% Custom list formatting
\setlist[itemize]{leftmargin=1.2em, itemsep=0em}

\begin{document}
	
	\begin{center}
		\Large\textbf{CE 397 Scientific Machine Learning}\\
		\large\textbf{Fall 2025}
	\end{center}
	
	\vspace{1em}
	\noindent
	\begin{tabularx}{\textwidth}{@{}p{0.2\textwidth}X@{}}
		\toprule
		\textbf{Instructor} & Krishna Kumar \\
		\textbf{Contact} & \href{mailto:krishnak@utexas.edu}{krishnak@utexas.edu} \\
		& 512-232-4406 \\
		\midrule
		\textbf{Office Hours} & M 2:00 - 3 PM and Wed 3 - 4 PM at ECJ 9.227 B \\
		\textbf{Lectures} & MW 12:30 PM - 2:00 PM at ECJ 3.402 \\
		\textbf{Textbooks} & A formal textbook is not required for this course. \\
		\textbf{References} & ``Scientific Machine Learning," [Draft Notes] Krishna Kumar, (2025). \\
		& ``Deep Learning," Aaron Courville, Ian Goodfellow, and Yoshua Bengio, (2015). \\
		\textbf{Course website} & Canvas \href{https://canvas.utexas.edu}{https://canvas.utexas.edu} \\
		\textbf{Prerequisites} & Basic programming skills in Python or a similar language\\
		& Familiarity with linear algebra, calculus, and basic probability theory\\
		& Some prior exposure to machine learning is recommended but not required\\
		\bottomrule
	\end{tabularx}
	
	\section*{Course Description}
	This course provides a rigorous introduction to Scientific Machine Learning (SciML), focusing on the development, analysis, and application of machine learning techniques for solving complex problems governed by ordinary and partial differential equations (ODEs/PDEs). Bridging numerical analysis, scientific computing, and deep learning, SciML offers novel computational paradigms for challenges where traditional methods face limitations, such as high-dimensional problems, inverse problems, and systems with incomplete physical knowledge.
	
	We will delve into the mathematical foundations underpinning modern SciML solvers. Key topics include Physics-Informed Neural Networks (PINNs), Neural Ordinary Differential Equations (NODEs), and Operator Learning frameworks (e.g., DeepONets, Fourier Neural Operators), which learn mappings between infinite-dimensional function spaces. The course will explore the theoretical basis for these methods, including function approximation theory in relevant spaces (e.g., Sobolev spaces), the role of automatic differentiation for computing derivatives and residuals, and the specific optimization challenges encountered when training physics-informed models.
	
	Emphasis will be placed on formulating differential equations as learning problems, analyzing the properties of different SciML architectures and loss functions, understanding techniques for enforcing boundary conditions, and evaluating the convergence and accuracy of the resulting solutions. We will also cover methods for uncertainty quantification, transfer learning approaches, and explore the discovery of governing equations from data. Practical implementation will be demonstrated using modern frameworks like PyTorch, JAX, and TensorFlow, enabling students to apply these advanced computational techniques to challenging scientific and engineering problems.
	
	\section*{Learning Outcomes}
	Upon successful completion of this course, students will be able to:
	\begin{itemize}
		\item Formulate scientific problems governed by ODEs/PDEs within machine learning frameworks; implement, train, and evaluate core methods like Physics-Informed Neural Networks (PINNs), Neural ODEs, and Operator Learning techniques, including appropriate handling of boundary conditions and sampling strategies.
		\item Analyze the mathematical underpinnings of SciML methods, including relevant function approximation theory (e.g., Universal Approximation Theorems in Sobolev spaces) and the specific optimization challenges (loss landscapes, convergence properties) associated with physics-informed training.
		\item Implement transfer learning approaches for domain adaptation in physical systems and apply few-shot learning techniques for rapid adaptation to new physical regimes with limited data.
		\item Implement and interpret methods for uncertainty quantification (e.g., Bayesian Neural Networks, Gaussian Processes) and data-driven discovery of governing differential equations (e.g., SINDy) within scientific applications.
		\item Critically assess the applicability, performance characteristics (accuracy, convergence, cost), and limitations of various SciML techniques in comparison to traditional numerical methods for solving differential equations.
		\item Effectively employ deep learning libraries (PyTorch, JAX, TensorFlow) leveraging automatic differentiation and best practices for reproducible SciML research and development, culminating in substantial project implementations.
	\end{itemize}
	
	\textbf{How Academic/Learning Goals Will Be Assessed:} These learning objectives will be assessed based on students' performance on homework assignments, comprehensive examination, and the final capstone project.
	
	\section*{Course Modules}
	The following 12 modules provide a comprehensive technical foundation in SciML:
	
	\paragraph{Module 1: Foundations of Machine Learning for Scientific Computing}
	Introduction to neural networks, optimization algorithms, and automatic differentiation for scientific applications. Covers Universal Approximation Theorem, gradient descent variants, and regularization techniques with emphasis on physics-based applications.
	
	\paragraph{Module 2: Mathematical Foundations for SciML}
	Essential mathematical concepts including function spaces (Hilbert, Banach, Sobolev), differential equations theory, numerical analysis fundamentals, and inverse problems formulation.
	
	\paragraph{Module 3: Physics-Informed Neural Networks (PINNs)}
	Comprehensive study of PINNs including mathematical formulation, boundary condition enforcement, training strategies, and advanced variants. Applications to forward and inverse ODE/PDE problems.
	
	\paragraph{Module 4: Neural Ordinary Differential Equations (Neural ODEs)}
	Continuous-depth neural networks through ODE parameterization, adjoint method for memory-efficient backpropagation, and applications to time-series modeling and latent dynamics learning.
	
	\paragraph{Module 5: Differentiable Programming and Physics Simulation}
	End-to-end differentiation through physics simulators, implementation of differentiable time-stepping schemes, and applications to inverse problems and parameter estimation.
	
	\paragraph{Module 6: Operator Learning I: DeepONet and Extensions}
	DeepONet architectures for learning function-to-function mappings, Universal Approximation Theorem for operators, and Physics-Informed DeepONet with multi-fidelity learning approaches.
	
	\paragraph{Module 7: Operator Learning II: Fourier Neural Operators and Advanced Methods}
	Fourier Neural Operators (FNOs) with spectral methods and advanced neural operators (Basis-to-Basis operator learning based on function encoders).
	
	\paragraph{Module 8: Graph Neural Networks (GNNs) for Scientific Simulation}
	Physical systems as graphs, message passing neural networks, symmetries and equivariance, with applications to particle-based systems and fluid dynamics.
	
	\paragraph{Module 9: Transfer Learning and Few-Shot Learning in SciML} Domain adaptation across physical regimes, meta-learning for new physical systems, multi-task learning, and active learning for optimal experimental design.
	
	\paragraph{Module 10: Transformers in Scientific Machine Learning} Adaptation of transformer architectures for scientific data, spatio-temporal modeling with attention mechanisms and applications.
	
	\paragraph{Module 11: Equation Discovery and Symbolic Methods}
	Sparse Identification of Nonlinear Dynamics (SINDy), symbolic regression, physics-guided discovery, and implementation workflows for discovering governing equations from data.
	
	\paragraph{Module 12: Advanced Generative Models and Uncertainty Quantification}
	Normalizing flows, Bayesian Neural Networks, Gaussian Processes, and physics-constrained VAEs for uncertainty quantification in scientific applications.
	
	\section*{Assessment}
	Your grade (G) will be computed based on the following formula:
	\[ G = 0.30H + 0.30E + 0.40FP \]
	where:
	\begin{itemize}
		\item \textbf{H (Homework - 30\%):} Weekly assignments combining theoretical analysis, mathematical derivations, and programming implementations
		\item \textbf{E (Exam - 30\%):} Comprehensive examination covering theoretical foundations and practical applications (Date TBC)
		\item \textbf{FP (Final Project - 40\%):} Comprehensive capstone project involving novel application, method development, or rigorous comparative analysis
	\end{itemize}
	
	\textbf{Graduate Research Philosophy:} This assessment structure reflects the realities of graduate-level scientific research, where success depends on sustained investigation, iterative development, and the ability to synthesize theory with implementation. Students develop expertise through deep engagement with challenging problems rather than performance on isolated examinations.
	
	\textbf{Attendance and Final Grade:} Class attendance is expected and contributes to course success, but is not directly factored into the numerical grade calculation.
	
	
	\subsection*{Assessment Details}
	
	\noindent\textbf{Homework Assignments (30\%):} Weekly assignments integrate theoretical analysis with practical implementation.
	
	\noindent\textbf{Exam (30\%):} A comprehensive examination covering:
	\begin{itemize}
		\item Theoretical foundations of SciML methods
		\item Mathematical analysis of neural network approximation theory
		\item Practical implementation considerations and optimization challenges
		\item Comparative analysis of different SciML approaches
		\item Date TBC
	\end{itemize}
	
	\noindent\textbf{Final Project (40\%):} A comprehensive capstone project where students will:
	\begin{itemize}
		\item Identify a novel scientific problem or extend existing methods
		\item Conduct literature review and problem formulation
		\item Implement advanced SciML techniques from multiple modules
		\item Compare with traditional numerical methods when applicable
		\item Include uncertainty quantification or transfer learning components
		\item Present results in conference-style paper and presentation
		\item Code must be reproducible and well-documented
	\end{itemize}
	
	\noindent\textbf{Resubmission Policy:} All assignments and projects may be resubmitted within 1 week of grading for full credit after addressing feedback. This encourages iterative learning, mastery of concepts, and mirrors the revision process central to research and publication.
	
	\noindent\textbf{Letter Grade Cutoffs:}
	\begin{itemize}
		\item A $\ge$ 94\%, A- $\ge$ 90\%
		\item B+ $\ge$ 87\%, B $\ge$ 84\%, B- $\ge$ 80\%
		\item C+ $\ge$ 77\%, C $\ge$ 74\%, C- $\ge$ 70\%
		\item D+ $\ge$ 67\%, D $\ge$ 64\%, D- $\ge$ 60\%
		\item F $<$ 60\%
	\end{itemize}
	
	The scale shown above represents minimum bounds. Adjustments may be made based on class performance.
	
	\section*{Grace Policy (Time-Bank)}
	Sometimes we have bad days, bad weeks, and bad semesters. In an effort to accommodate any unexpected, unfortunate personal crisis, I have built "time banks" into our course. You do not have to utilize this policy, but if you find yourself struggling with unexpected personal events, I encourage you to contact me as soon as possible.
	
	\textit{Each student will get a total of 7 days of NO QUESTIONS ASKED extension for homework and project deadlines. It is up to each student to use those days at once or for multiple assignments. These will be accounted for in increments of 1-day and should be requested through email (indicating proposed submission date) 24hrs before the original deadline.}
	
	\section*{Course Attendance}
	Students are \textbf{expected to attend all class periods}. Those who fail to attend class regularly are inviting scholastic difficulty. Students are responsible for material covered in class, even if absent for authorized activities. \textit{It is crucial to communicate with your project team members; you are responsible for letting both the instructor and your team know if you cannot make it to class.}
	\textbf{Excused Absence:} If you plan to miss class due to religious observance, please provide at least two weeks advance notice. For further information: \href{http://catalog.utexas.edu/general-information/academic-policies-and-procedures/attendance/}{UT Attendance Policy}
	
	\section*{Computing Resources}
	\textbf{Cloud Access:} The class will use Google Colab for all assignments, in-class materials, and projects. Access directly at \href{https://colab.research.google.com/notebooks/intro.ipynb}{Google Colab}. Your UT Email account provides direct access. Students may also use \href{https://repl.it/}{Repl.it} as an alternative online IDE.
	\textbf{High-Performance Computing:} For larger projects, students will have access to university computing clusters through TACC (Texas Advanced Computing Center). Training sessions will be provided.
	
	\section*{Academic Integrity}
	Each student is expected to abide by the University of Texas Honor Code: "\textit{As a student of The University of Texas at Austin, I shall abide by the core values of the University and uphold academic integrity.}" \textbf{Plagiarism is taken very seriously at UT.} If you use words or ideas that are not your own, you must cite your sources. Copying solutions from online sources (Chegg, CourseHero) is considered cheating.
	
	You are \textbf{welcome and encouraged to use AI/Large Language Models} (Gemini, Claude, etc.) for your assignments and projects. Please see acceptable AI tool use: \href{https://security.utexas.edu/ai-tools}{UT AI Tools Policy}. However, you must understand and be able to explain any code or solutions you submit.
	
	\subsection*{Collaboration Policy}
	Students are encouraged to discuss course topics and collaborate on understanding concepts. However:
	\begin{itemize}
		\item All submitted work must be your own individual effort
		\item You may discuss approaches but must implement solutions independently  
		\item Identical code or solutions are not acceptable
		\item Proper attribution must be given for any external resources used
	\end{itemize}
	
	\subsection*{Sharing of Course Materials}
	No materials used in this class, including, but not limited to, lecture hand-outs, videos, assessments (quizzes, exams, papers, projects, homework assignments), in-class materials, review sheets, and additional problem sets, may be shared online or with anyone outside of the class unless you have my explicit, written permission. Unauthorized sharing of materials promotes cheating. It is a violation of the University's Student Honor Code and an act of academic dishonesty. I am well aware of the sites used for sharing materials, and any materials found online that are associated with you, or any suspected unauthorized sharing of materials, will be reported to Student Conduct and Academic Integrity in the Office of the Dean of Students. These reports can result in sanctions, including failure in the course.
	
	\subsection*{Confidentiality of Class Recordings}
	Class recordings are reserved only for students in this class for educational purposes and are protected under FERPA. The recordings should not be shared outside the class in any form. Violation of this restriction by a student could lead to Student Misconduct proceedings.
	
	\section*{University Resources and Policies}
	
	\subsection*{University Health Services}
	Important safety information and University health services resources can be found here:
	
	\href{https://www.healthyhorns.utexas.edu/index.html}{https://www.healthyhorns.utexas.edu/index.html}
	
	\subsection*{Disability \& Access (D\&A)}
	The university is committed to creating an accessible learning environment consistent with university policy and federal and state law. Please let your instructors know if you experience any barriers to learning so they can work with you to ensure you have equal opportunity to participate fully in your courses. If you are a student with a disability, or think you may have a disability, and need accommodations please contact Disability \& Access (D\&A). Please refer to the \href{https://disability.utexas.edu/}{D\&A website} for more information. If you are already registered with D\&A, please deliver your accommodation letter to your instructors as early as possible in the semester so you can discuss together your approved accommodations and needs in your courses.
	
	\subsection*{Course and Instructor Evaluation}
	An evaluation of the course and instructor will be conducted at the end of the semester using the approved UT Course/Instructor evaluation forms.
	
	\subsection*{Undergraduate Student Drop Policy}
	From the 1\textsuperscript{st} through the 12\textsuperscript{th} class day (4\textsuperscript{th} class day in the summer sessions), an undergraduate student may drop a class through the registration system. From the 13\textsuperscript{th} (5\textsuperscript{th} class day in the summer sessions) through the university's academic drop deadline, a student may Q drop a class with approval from the Dean, and departmental advisor.
	
	\subsection*{Emergency Preparedness Plan}
	Emergency Preparedness means being ready. It takes an effort by all of us to create and sustain an effective emergency preparedness system. You are your own best first responder. Please use \href{https://preparedness.utexas.edu/welcome-emergency-preparedness}{https://preparedness.utexas.edu/welcome-emergency-preparedness} as a resource to better understand emergency preparedness at the university, and how you can become part of and contribute to the preparedness community. To monitor emergency communications for specific instructions, go to \href{https://utexas.edu/emergency}{https://utexas.edu/emergency}. To report an issue (non emergency) call 512-471-4441. In case of emergency, call 911.
	
	\subsection*{Counseling and Mental Health Center (CMHC)}
	Support is available for academic stress, anxiety, depression, and other challenges. Contact: \href{http://www.cmhc.utexas.edu/individualcounseling.html}{CMHC Individual Counseling}
	
	\subsection*{Other Resources}
	\begin{itemize}
		\item \textbf{Sanger Learning Center:} Tutoring and academic support - \href{http://www.utexas.edu/ugs/slc}{utexas.edu/ugs/slc} | 512-471-3614
		\item \textbf{Undergraduate Writing Center:} \href{http://uwc.utexas.edu/}{uwc.utexas.edu}
		\item \textbf{Libraries:} \href{http://www.lib.utexas.edu/}{lib.utexas.edu}
		\item \textbf{Student Emergency Services:} \href{http://deanofstudents.utexas.edu/emergency/}{deanofstudents.utexas.edu/emergency}
		\item \textbf{Behavior Concerns Advice Line (BCAL):} 512-232-5050 (anonymous reporting)
	\end{itemize}
	
	\subsection*{Title IX Reporting}
	I am a Responsible Employee and must report any Title IX-related incidents disclosed to me. For confidential support without official reporting, email \href{mailto:advocate@austin.utexas.edu}{advocate@austin.utexas.edu}. More information: \href{http://titleix.utexas.edu/}{titleix.utexas.edu}
	
	\subsection*{Emergency Evacuation}
	Familiarize yourself with exit routes. If you need assistance during evacuation, inform me in writing during the first week. Follow faculty instructions during evacuations. More information: \href{https://preparedness.utexas.edu/safety/emergency-terms}{UT Emergency Procedures}
	
	\subsection*{Statement on Learning Success}
	Your success is important. If aspects of this course prevent you from learning or exclude you, please let me know. We can develop strategies together. I encourage you to use UT's student resources.
	
	\subsection*{Inclusive Classroom}
	This classroom is a place where you will be treated with respect. I welcome individuals of all backgrounds, beliefs, identities, and abilities. All members are expected to contribute to a respectful, welcoming, and inclusive environment.
	
	\subsection*{BeVocal}
	BeVocal is a university-wide initiative to promote the idea that individual Longhorns have the power to prevent high-risk behavior and harm. At UT Austin all Longhorns have the power to intervene and reduce harm. To learn more about BeVocal and how you can help to build a culture of care on campus, visit \href{https://wellnessnetwork.utexas.edu/BeVocal}{wellnessnetwork.utexas.edu/BeVocal}.
	
	\subsection*{Important Safety Information}
	If you have concerns about the safety or behavior of fellow students, TAs or Professors, call BCAL (the Behavior Concerns Advice Line): 512-232-5050. Your call can be anonymous. If something doesn't feel right — it probably isn't. Trust your instincts and share your concerns.
	
	The following recommendations regarding emergency evacuation from the Office of Campus Safety and Security, 512-471-5767, \href{http://www.utexas.edu/safety/}{utexas.edu/safety}:
	
	Occupants of buildings on The University of Texas at Austin campus are required to evacuate buildings when a fire alarm is activated. Alarm activation or announcement requires exiting and assembling outside.
	\begin{itemize}
		\item Familiarize yourself with all exit doors of each classroom and building you may occupy. Remember that the nearest exit door may not be the one you used when entering the building.
		\item Students requiring assistance in evacuation shall inform their instructor in writing during the first week of class.
		\item In the event of an evacuation, follow the instruction of faculty or class instructors. Do not re-enter a building unless given instructions by the following: Austin Fire Department, The University of Texas at Austin Police Department, or Fire Prevention Services office.
		\item Link to information regarding emergency evacuation routes and emergency procedures can be found at \href{https://preparedness.utexas.edu/safety/emergency-terms}{preparedness.utexas.edu/safety/emergency-terms}
	\end{itemize}
	
	\section*{Relationship of the Course to Student Outcomes}
	This course is designed to achieve the ABET Student Outcomes marked with an "X" in the first column.
	
	\noindent\begin{tabularx}{\textwidth}{@{}p{0.1\textwidth}X@{}}
		\toprule
		\textbf{} & \textbf{Student Outcomes} \\
		\midrule
		X & 1. An ability to identify, formulate, and solve complex engineering problems by applying principles of engineering, science, and mathematics \\
		
		& 2. An ability to apply engineering design to produce solutions that meet specified needs with consideration of public health, safety, and welfare, as well as global, cultural, social, environmental, and economic factors \\
		
		& 3. An ability to communicate effectively with a range of audiences \\
		
		X & 4. An ability to recognize ethical and professional responsibilities in engineering situations and make informed judgments, which must consider the impact of engineering solutions in global, economic, environmental, and societal contexts \\
		
		& 5. An ability to function effectively on a team whose members together provide leadership, create a collaborative and inclusive environment, establish goals, plan tasks, and meet objectives \\
		
		X & 6. An ability to develop and conduct appropriate experimentation, analyze and interpret data, and use engineering judgment to draw conclusions \\
		
		X & 7. An ability to acquire and apply new knowledge as needed, using appropriate learning strategies \\
		\bottomrule
	\end{tabularx}
	
	\subsection*{General Information}
	All other university policies not explicitly included on this syllabus can be found on the General Information Catalog: \href{http://catalog.utexas.edu/general-information/}{catalog.utexas.edu/general-information}.
	
\end{document}